	\documentclass[a4paper]{article}
\usepackage{amsmath}
\usepackage{xspace}
\usepackage{times}

\newcommand{\intinfty}{\int^{+\infty}_{-\infty}\xspace}
\newcommand{\abs}[1]{\left|{#1}\right|\xspace}
\newcommand{\ftf}[1]{{\mathcal{F}\left[#1\right]\xspace}}
\newcommand{\iftf}[1]{{\mathcal{F}^{-1}\left[#1\right]\xspace}}
\newcommand{\ftt}[1]{{\tilde{#1}\xspace}}
\newcommand{\comb}[1]{{\mathrm{comb}\left( #1 \right)}\xspace}
\newcommand{\Dz}{{\Delta z}\xspace}
\newcommand{\Dk}{{\Delta k}\xspace}
\newcommand{\cst}[1]{{\mathrm{#1}}\xspace}
\newcommand{\vcvs}{{\textit{vice versa}\xspace}}

\title{Quick reference of Fourier transform for Fourier domain optical coherence tomography}
\author{%
	Yoshiaki Yasuno\\
	\textit{Computational Optics Group at the University of Tsukuba}\\
	$<$yoshiaki.yasuno@cog-labs.org$>$
}
\date{Ver. 1.1.1 released on 2025-03-24}
\begin{document}
\maketitle
\tableofcontents

\section{Preface}
This short memorandum provides a quick reference for Fourier analysis.
Among various textbooks on the subject, this note is distinguished by its focus on Fourier-domain optical coherence tomography (FD-OCT).
FD-OCT is typically formulated using the Fourier pair $(z, k)$, where $z$ represents depth position and $k$ denotes angular wavenumber.
In contrast, most Fourier analysis textbooks use the Fourier pair $(t,\nu)$, where $t$ represents time and $\nu$ is frequency.
Since the frequency $\nu$ for $t$ corresponds to the oscillation wavenumber $\kappa$ for $z$, the standard textbook notation cannot be directly applicable to FD-OCT.
Although converting between the $(z,\kappa)$ and $(z,k)$ formulations is conceptually straightforward, it can sometimes be intricate and prone to inconsistencies.
This reference provides a set of consistent formulas based on the $(z, k)$ Fourier pair, and hence, can serve as a toolbox for learning and developing FD-OCT theory.
Additionally, the same formulas can be applied to  $(t, \omega)$ Fourier analysis by substituting $z$ with $t$ and $k$ with $\omega$, where $t$ is time and $\omega$ is angular frequency.
Therefore, this reference can also be used for constructing theories based on the $(t, \omega)$ formulation.

\section{Notations}
In this memorandum, the following notations are used:
%\begin{itemize}

$\lambda$ denotes wavelength, typically measured in meters (m) in the SI system.

$k$ is the angular wavenumber, defined as $k = 2\pi/\lambda$.
In FD-OCT formulations, the spectral domain variable is typically $k$.

$\kappa$ is the oscillation wavenumber given by $\kappa = k/\pi= 1/\lambda$.
It is rarely used in FD-OCT formulations.

$z$ represents the double-path axial optical depth and is the Fourier pair of  $k$.
The single-path physical depth is given by $z/2n$, where $n$ is the refractive index.

$\ftf{\quad}$ and $\iftf{\quad}$ denote the Fourier transform and inverse Fourier transform, respectively.

$f(z)$ and $\ftt{f}(k)$ represent a function of $z$ and its Fourier spectrum in the $k$-domain, respectively.
In general, any function with a tilde ($\tilde{\quad}$) represents the Fourier spectrum of the corresponding function without the tilde.

The operators $*$ and $\otimes$ represent convolution and correlation, respectively.

The superscript $*$ denotes the complex conjugate.

\section{Definition of Fourier transform}
\subsection{Fourier transform}
The Fourier transform that transforms a function of $z$ to its $k$ spectrum is defines as
follows.
\begin{equation}
	\ftt{f}(k) = \ftf{f(z)} \equiv \int^{+\infty}_{-\infty} f(z) e^{-ikz}dz,
\end{equation}
where $\ftf{\quad}$ represents Fourier transform.

\subsection{Inverse Fourier transform}
The inverse Fourier transform from the $k$ domain to the $z$ domain is defined as
\begin{equation}
	{f}(z) = \iftf{\ftt{f}(k)} \equiv \frac{1}{2\pi} \int^{+\infty}_{-\infty} \ftt{f}(k) e^{ikz}dk,
\end{equation}
where $\ftf{\quad}$ represents inverse Fourier transform.
$1/2 \pi$ is a constant to make the Fourier transform and the inverse Fourier transform as a reversible pair.

\subsection{Euler's formula and cos-sin-form of Fourier transform}
Euler's formula can decompose exponential function into cosine and sine functions as
\begin{equation}
	e^{i\theta} = \cos \theta + i \sin\theta,
\end{equation}
and
\begin{equation}
	e^{-i\theta} = \cos \theta - i \sin\theta,
\end{equation}
where $\theta$ is a phase value or function.

Using Euler's formula, the Fourier and the inverse Fourier transforms can be written in the cos-and-sin form as
\begin{equation}
	\ftt{f}(k) = \int^{+\infty}_{-\infty} f(z) e^{-ikz}dz =
	 \int^{+\infty}_{-\infty} f(z) \cos(kz) \,\, dz
	-i\int^{+\infty}_{-\infty} f(z) \sin(kz) \,\, dz,
\end{equation}
and
\begin{equation}
	f(z) = \frac{1}{2\pi} \int^{+\infty}_{-\infty} \ftt{f}(k) e^{ikz} dk =
	\frac{1}{2\pi} \left[\int^{+\infty}_{-\infty} \ftt{f}(k) \cos(kz) dk
	+i \int^{+\infty}_{-\infty} \ftt{f}(k) \sin(kz) dk\right].
\end{equation}

\section{Convolution and convolution theorem}
\subsection{Convolution}
The convolution of two functions $f(z)$ and $g(z)$ is defines as
\begin{equation}
	f(z)*g(z) \equiv \int_{-\infty}^{+\infty} f\left(z'\right)g\left(z-z'\right	)dz',
\end{equation}
where $*$ denotes convolution.
\subsection{$z$-domain convolution theory}
The Fourier transform of convolution of functions in $z$ becomes a multiplication of the Fourier transformations of these functions.
\begin{equation}
	\ftf{f(z)*g(z)} = \ftt{f}(k) \ftt{g}(k).
\end{equation}
Similarly,
\begin{equation}
	\iftf{\ftt{f}(k)\ftt{g}(k)} = f(z)*g(z).
\end{equation}
This relationship is known as convolution theory.

\subsection{$k$-domain convolution theory}
The convolution theory for the convolution of $k$-domain functions is
\begin{equation}
	\iftf{\ftt{f}(k)*\ftt{g}(k)} = 2\pi f(z) g(z).
\end{equation}
Similarly,
\begin{equation}
	\ftf{{f}(z)g(z)} = \frac{1}{2\pi}\ftt{f}(k)*\ftt{g}(k).
\end{equation}
\subsection{Multiple-convolution and associative law of convolution}
By sequentially applying convolution theory, it can be extended into three or more functions as
\begin{equation}
	\ftf{\left\{f(z) * g(z)\right\} * h(z)}
	 = \ftf{f(z) * g(z)}\ftt{h}(k) = \ftt{f}(k)\ftt{g}(k)\ftt{h}(k).
\end{equation}
It implies the associative rule of the convolution
\begin{equation}
	\left[f(z)*g(z)\right]*h(z)=f(z)*\left[g(z)*h(z)\right].
\end{equation}

Similarly, the $k$-domain convolution theory of three functions becomes
\begin{equation}
	\iftf{\left\{\ftt{k}(k)* \ftt{g}(k)\right\}*\ftt{h}(k)}
	= \frac{1}{2\pi}\iftf{\ftt{f}(k)*\ftt{g}(k)}h(z) = \frac{1}{\left(2\pi\right)^2}f(z)g(z)h(z).
\end{equation}
Note that, for multiple convolution of $n$ functions, the coefficient in the right hand side of the equation becomes $1/(2\pi)^{(n-1)}$.

The associative law of convolution of $k$-domain function is the same with that of $z$-domain function as
\begin{equation}
	\left[\ftt{f}(k)*\ftt{g}(k)\right]*\ftt{h}(k) = \ftt{f}(k) * \left[\ftt{g}(k)*\ftt{h}(k)\right].
\end{equation}

\subsection{Commutative law of convolution}
The commutative law stands for convolution as
\begin{equation}
	f(z) * g(z) = g(z) * f(z),
\end{equation}
and
\begin{equation}
	\ftt{f}(k) * \ftt{g}(k) = \ftt{g}(k) * \ftt{f}(k).
\end{equation}

\section{Delta-function and shift law}
\subsection{Delta function and its Fourier transform}
The Dirac delta function, or ``delta function'' in short, is defined as
\begin{equation}
	\delta\left( x \right) =
	\left\{
		\begin{array}{ccl}
			0& \mathrm{for} & x \neq 0\\
			+\infty& \mathrm{for} & x = 0	
		\end{array}
	\right.,
\end{equation}
where $x$ is a general variable, and
\begin{equation}
	\intinfty \delta\left( x \right)\,\, dx = 1.
\end{equation}

The Fourier transform of a delta-function is a complex harmonic function whose frequency is defined by the position of the delta function as
\begin{equation}
	\ftf{\delta\left(z-\mathrm{z_1}\right)} = e^{-i \mathrm{z_1} k}
	\qquad\Leftrightarrow\qquad	
	\iftf{e^{-i \mathrm{z_1} k}} = \delta\left(z-\mathrm{z_1}\right),
\end{equation}
where $\mathrm{z_1}$ is a constant representing the position of the $z$-domain delta function.

Similarly, for $k$-domain delta function
\begin{equation}
	\label{eq:deltaK}
	\iftf{\delta\left(k-\mathrm{k_1}\right)} = \frac{1}{2\pi} e^{i \mathrm{k_1} z}
	\qquad\Leftrightarrow\qquad
	\ftf{e^{i \mathrm{k_1} z}} = 2\pi\delta\left(k-\mathrm{k_1}\right),
\end{equation}
where $\mathrm{k_1}$ is a constant representing the position of the $k$-domain delta function.

\subsection{Shift law of Fourier transform}
By using the convolution theorem and the Fourier transform of delta-function, the Fourier transform of a shifted function can be obtained as	
\begin{equation}
	\ftf{ f(z) * \delta\left(z-\mathrm{z_1}\right)} = \ftt{f}(k) e^{-i \mathrm{z_1} k}
\end{equation}
Its $k$-domain version is
\begin{equation}
	\iftf{\ftt{f}(k)* \delta\left(k- \mathrm{k_1}\right)} = f(z) e^{i \mathrm{k_1} z}.
\end{equation}
Shift law indicates that if the signal is shifted, a linear phase slope is added to its spectrum.
Similarly, if the spectrum of a function is shifted, a linear phase slope is added to the function.

\section{Comb function and its distributional formulation}
\subsection{Comb-function and Fourier transform}
A comb-function is defined as a train of delta-functions as
\begin{equation}
	\comb{ \frac{z}{\Dz} }
	\equiv \sum_{n = -\infty}^{+\infty} \delta\left( z- n\Delta \right),
\end{equation}
where $\Delta z$ is a period of the delta-functions train, i.e., the separation of two adjacent delta-functions.

The Fourier transform of the comb function becomes
\begin{equation}
	\label{eq:ftCombZ}
	\ftf{ \comb{\frac{z}{\Dz}} }  = \frac{2\pi}{\Dz} \comb{k\frac{\Dz}{2 \pi}}.
\end{equation}
Namely, the Fourier transform of a comb function with a period of $\Delta z$ is a comb function with a period of $2\pi/\Dz$.
It is noteworthy that shorter the period in the $z$ domain results in the longer the period in the $k$ domain and \vcvs.

The inverse Fourier transform of a $k$-domain comb function is
\begin{equation}
	\iftf{\comb{\frac{k}{\Dk}}} = \frac{1}{\Dk} \comb{z\frac{\Dk}{2\pi}}.
\end{equation}
Namely, the inverse Fourier transform of a comb function with a period of $\Dk$ is a comb function with a period of $2\pi/\Dk$.
Similar to the former case, shorter the period in the $k$ domain results in the longer the period in the $z$ domain and \vcvs.

\subsection{Distributional form of comb-function}
By using Poisson summation formula, the comb-function can be expressed in its distribution form as
\begin{equation}
	\comb{\frac{z}{\Dz}} 
	= \sum_{n = -\infty}^{+\infty} \delta\left( z-n\Dz \right)
	= \frac{1}{\Dz} \sum_{m=-\infty}^{+\infty} \exp\left(i2\pi \frac{m}{\Dz} z\right)
\end{equation}
The Fourier transform of a comb-function [Eq.\@ (\ref{eq:ftCombZ})] can be derived by using this distributional form and the formula of the Fourier transform of delta-function [Eq.\@ (\ref{eq:deltaK})].

\section{Auto-correlation and Wiener-Khintchine’s theorem}
{\bf NOTE!:}
This section may be inconsistent for complex functions, and may consist some errors.
Don't blindly trust this section.
The material will be corrected soon.
\subsection{Cross- and auto-correlation functions}
The cross-correlation function of two functions $f(z)$ and $g(z)$ is defines as
\begin{equation}
	f(z) \otimes g(z) \equiv \int_{-\infty}^{+\infty} f(z') g(z'-z)\, dz',
\end{equation}
where $\otimes$ represents correlation.

The correlation function between a same function is called as auto-correlation, and is written as
\begin{equation}
	f(z) \otimes f(z)  = \int_{-\infty}^{+\infty} f(z') f(z'-z)\, dz'.
\end{equation}

The correlation and convolution are related as 
\begin{equation}
	f(z) \otimes g(z) = f(z) * g(-z).
\end{equation}
This can be proven using $g'(z) \equiv g(-z)$ as
\begin{equation}
	f(z) \otimes g(z) = \int_{-\infty}^{+\infty} f(z') g(z'-z)\, dz'
	= \int_{-\infty}^{+\infty} f(z') g'(z-z')\, dz' = f(z) * g(-z).
\end{equation}

% YAS: Clarify the relationship between the correlation function and convolution here.

\subsection{Wiener-Khintchine’s theorem}
The inverse Fourier transform of the power spectrum of a function is the auto-correlation of the function;
\begin{equation}
	\iftf{ \left|\ftt{f}(k)\right|^2} = f(z) \otimes f(z).
\end{equation}
This relation between the power spectrum and the auto-correlation is known as Wiener-Khintchine’s theorem.

\section{Scaling and mirroring of function}
\subsection{Scaling property of Fourier transform}
The Fourier transform of a function scaled in $z$ is
\begin{equation}
	\label{eq:scale-z}
	\ftf{ f(\cst{a} z) } = \frac{1}{\left|\cst{a}\right|} \ftt{f}\left(\frac{k}{\cst{a}}\right),
\end{equation}
where $\cst{a}$ is a constant scaling the function along $z$.
Namely, $f(\cst{a}z)$ is $1/\cst{a}$-times narrower than $f(z)$. 

Similarly,
\begin{equation}
	\label{eq:scale-k}
	\iftf{ \ftt{f}(\cst{b} k) } = \frac{1}{\left|\cst{b}\right|} \ftt{f}\left(\frac{z}{\cst{b}}\right),
\end{equation}
where $\cst{b}$ is another scaling factor.

\subsection{Fourier transform of a mirrored signal}
The Fourier transform of a mirrored signal is obtained as
\begin{equation}
	\ftf{ f(-z) } = \ftt{f} (-k) 
	\qquad \Leftrightarrow \qquad
	\iftf{ \ftt{f}(-k) } = f(-z),
\end{equation}
where $f(-z)$ is the mirrored signal of $f(z)$.

Note that this is the special case of the Fourier transform of the scaled function [Eqs.\@ (\ref{eq:scale-z}) and (\ref{eq:scale-k})] with the scaling factor of -1.

\section{Fourier transform of real function and complex conjugate}
\subsection{Fourier transform of real function}
The Fourier transform spectrum of a real function $r(z)$ is a mirror-conjugate-symmetric function, namely,
\begin{equation}
	\ftt{r}(k) = \ftt{r^*}(-k),
\end{equation}
where $\ftt{r}(k)$ is $\ftf{r(z)}$, and the superscript of $*$ denotes complex conjugate.
Note that $\ftt{r}(k)$ is not necessarily a real function.
Namely, the Fourier transform of a real function can be a complex function.
%(i.e., $\ftf{r(z)}$) is not necessarily a real function but can be a complex function.
%, and $r(z)$ is a real functions.

Similarly, for a real $k$-domain function, i.e., a spectrum, $\ftt{s}(k)$,
\begin{equation}
	s(z) = s^*(-z),
\end{equation}
where $s(k)$ is $\iftf{\ftt{s}(k)}$, and $s(k)$ is a real functions.

This relation is exemplified by FD-OCT.
A spectral interferogram is a real function, and hence its Fourier transform in $z$-domain is mirror-conjugate-symmetric.
(Note that it is not true if additional phase is applied to the spectral interferogram, as it is frequently done to correct dispersion.)

\subsection{Fourier transform of complex conjugate}
The Fourier transform of the complex conjugate of a function becomes a mirrored complex conjugate of the original Fourier tansform [i.e., spectrum $\ftt{f}(k)$] as
\begin{equation}
	\ftf{f^*(z)} = \ftt{f}^*(-k),
\end{equation}
where superscript of $*$ denotes complex conjugate.

Similarly,
\begin{equation}
	\iftf{\ftt{f}^*(k)} = f^*(-z).
\end{equation}

%% The following point should be explicitely proven.
This relationship can be derived by resolving $f(z)$ into two real functions as $f(z) = f_r(z) + i f_i(z)$, where both $f_r(z)$ and $f_i(z)$ are real functions, Fourier transform it, and then applying the rule for the Fourier transform of a real function.

\section{Parseval’s theorem}
Parseval’s theorem is the law of energy conservation during Fourier transform, and is expressed as 
\begin{equation}
	\intinfty \abs{f(z)}^2 \, dz
	= \frac{1}{2\pi} \intinfty \abs{\ftt{f}(k)}^2  \, dk.
\end{equation}

\section*{Outlooks}
The following topics are going to be added.
\begin{itemize}
	\item Following Window functions and is Fourier transform
	\begin{itemize}
		\item Rectangle function
		\item Gauss function
		\item Hanning function
		\item Hamming function
	\end{itemize}
	\item The relationship between convolution and correlation functions.
	\item Hilbert transform as Fourier filter
\end{itemize}

\section*{Release log}
\begin{itemize}
\item 2025-04-15 (Tue): [ver1.3pre01] Dirac delta function and sin-and-cos form of Fourier transform are added by Yasuno.
\item 2025-03-27 (Thu): [Ver1.2pre01] Reformatted and several interpretations of equations added by Yasuno.
\item 2015-06-02 (Tue): [Ver1.1] ``Comb function'' was added by Yasuno.	
\item 2014-08-21 (Thu): [Ver1.0] The first pre-view release by Yasuno.
\end{itemize}
\end{document}