\documentclass{article}[10pt,a4paper]
\usepackage{xspace,amssymb, times}
%\newcommand{\ft}[1]{\mathfrak{F}\left[#1\right]\xspace}
\newcommand{\ft}[1]{\mathcal{F}\left[{#1}\right]\xspace}
\newcommand{\ift}[1]{\mathcal{F}^{-1}\left[{#1}\right]\xspace}
\newcommand{\sexp}[1]{e^{#1}\xspace}
\newcommand{\intinfty}{\int^{+\infty}_{-\infty}\xspace}
\newcommand{\abs}[1]{\left|{#1}\right|\xspace}

\newcommand{\ftf}[1]{{\mathcal{F}\left[#1\right]\xspace}}
\newcommand{\iftf}[1]{{\mathcal{F}^{-1}\left[#1\right]\xspace}}
\newcommand{\ftt}[1]{{\tilde{#1}\xspace}}
\newcommand{\comb}[1]{{\mathrm{comb}\left( #1 \right)}\xspace}
\newcommand{\Dz}{{\Delta z}\xspace}
\newcommand{\Dk}{{\Delta k}\xspace}
\newcommand{\cst}[1]{{\mathrm{#1}}\xspace}


\title{Quick Reference of Fourier Transform for Fourier Domain Optical
Coherence Tomography}

\author{Yoshiaki Yasuno}
\date{2014-08-11}
\begin{document}
\maketitle
\tableofcontents

\section{Introduction}
This short note aims at providing a quick reference of Fourier transform, its basic rules, and some typical Fourier transform pairs.
Among several such textbooks, this short note is characterized by its focus toward Fourier domain optical coherence tomography (FD-OCT).

\section{Notations}
In this short note, the following notations are utilized.

$\lambda$ denotes wavelength and its unit is typically m in SI system.

$k$ is an {\em angular} wavenumber. $k = 2\pi/\lambda$.
In the formulation of FD-OCT, a variable of spectral domain is typically
$k$.

$\kappa$ is an {\em oscillation} wavenumber; $\kappa = k/2\pi =
1/\lambda$. $\kappa$ is rarely used in formulation of FD-OCT.

$z$ is a {\em double-path} axial optical depth, and is the Fourier pair
of $k$.
The single path physical depth is then represented as $z/2n$ with $n$ as a refractive index.

\section{Definition of Fourier transform}

\subsubsection*{Fourier transform}
\noindent
The Fourier transform that transforms a function of $z$ to its $k$
spectrum is defines as follows.
\begin{equation}
 \tilde{f}(k) = \ft{f(z)} = \intinfty f(z) \sexp{-ikz} dz,
\end{equation}
where $\ft{\quad}$ represents Fourier transform.

\subsubsection*{Inverse Fourier transform}
%\noindent
%\textbf{Inverse Fourier transform}\\
The inverse Fourier transform ($k$ to $z$) is defined as
\begin{equation}
 f(z) = \ift{\tilde{f}(k)} = \frac{1}{2 \pi} \tilde{f}(k) \sexp{ikz} dk,
\end{equation}
where $\ft{\quad}$ represents inverse Fourier transform.
$1/2\pi$ is a constant to make the Fourier transform and the inverse
Fourier transform as a reversible pair.

\section{Convolution and convolution theorem}
\subsubsection*{Convolution}
The convolution of two functions $f(z)$ and $g(z)$ is defines as
\begin{equation}
 f(z)*g(z) \equiv \intinfty f(z') g(z-z') dz',
\end{equation}
where $*$ denotes convolution.

\subsubsection*{$z$-domain convolution theory}
The Fourier transform of convolution of functions in $z$ becomes a
multiplication of the Fourier transformations of these functions.
\begin{equation}
 \ft{f(z)*g(z)} = \tilde{f}(k) \tilde{g}(k).
\end{equation}
Similarly,
\begin{equation}
 \ift{\tilde{f}(k)*\tilde{g}(k)} = f(z)*g(z).
\end{equation}
This relationship is known as convolution theory.

\subsubsection*{$k$-domain convolution theory}
The convolution theory of the convolution of $k$ domain functions is
\begin{equation}
 \ift{\tilde{f}*(k)*\tilde{g}(k)} = 2\pi f(z)g(z).
\end{equation}
Similarly,
\begin{equation}
 \ft{f(z)g(z)} = \frac{1}{2\pi} \tilde{f}(k) *\tilde{g}(k).
\end{equation}

\subsubsection*{Multiple-convolution and associative law of convolution}
By sequentially applying convolution theory, it can be extended into
three or more functions as
\begin{equation}
 \ft{\left\{f(z)*g(z)\right\}*h(z)} 
  = \ft{f(z)*g(z)} \tilde{h}(k) 
  = \tilde{f}(k) \tilde{g}(k) \tilde{h}(k).
\end{equation}
It implies the associative rule of the convolution
\begin{equation}
 \left[f(z)*g(z)\right]*h(z) = f(z)*\left[g(z)*h(z)\right]
\end{equation}

Similarly, the $k$-domain convolution theory of three functions becomes
\begin{equation}
 \ift{\left\{\tilde{f}(k) * \tilde{g}(k) \right\}*\tilde{h}(k)} 
  = \frac{1}{2\pi}\ift{\tilde{f}(k)* \tilde{g}(k)} {h}(z) 
  = \frac{1}{(2\pi)^2} f(z) g(z) h(z).
\end{equation}
Note that, for multiple convolution of $n$ functions, the coefficient in
the right hand side of the equation becomes $1/(2 \pi)^{(n-1)}$.

The associative law of convolution of $k$-domain function is the same
with that of $z$-domain function as
\begin{equation}
 \left[\tilde{f}(k)*\tilde{g}(k)\right]*\tilde{h}(k) =
  \tilde{f}(k)*\left[\tilde{g}(k)*\tilde{h}(k)\right].
\end{equation}

\subsubsection*{Commutative law of convolution}
The commutative law stands for convolution as
\begin{equation}
 f(z)*g(z) = g(z) * f(z),
\end{equation}
and 
\begin{equation}
 \tilde{f}(k)* \tilde{g}(k) = \tilde{g}(k)* \tilde{f}(k).
\end{equation}

\section{Delta-function and shift law}
\subsubsection*{Delta-function and its Fourier transform}
The Fourier transform of a delta-function is a complex harmonic
function whose frequency is defined by the position of the delta
function;
\begin{equation}
 \ft{\delta(z-z_0)} = \sexp{-i k z_0}
  \quad \Leftrightarrow \quad
  \ift{\sexp{-ikz_0}} = \delta(z-z_0).
\end{equation}
Similarly, for $k$-domain delta function
\begin{equation}
 \ift{\delta(k-k_0)} = \frac{1}{2\pi} \sexp{i k_0 z}
  \quad \Leftrightarrow \quad
  \ft{\sexp{ik_0 z}} = 2\pi \delta(k-k_0).
\end{equation}

\subsubsection*{Shift law of Fourier transform}
By using the convolution theorem and the Fourier transform of
delta-function, the Fourier transform of a shifted function can be
obtained as
\begin{eqnarray}
 \ft{ f(z) * \delta(z-z_0)} = \tilde{f}(k) \sexp{-ikz_0}
\end{eqnarray}
Its $k$-domain version is
\begin{equation}
 \ift{\tilde{f}(k) * \delta(k-k_0)} = f(z) \sexp{ik_0 z}
\end{equation}

Shift law indicates that if the signal is shifted, a linear phase slope is
added to its spectrum.
Similarly, if the spectrum of a function is shifted, a linear phase
slope is added to the function.

\section{Auto-correlation and Wiener-Khintchine's theorem}
\subsubsection*{Cross- and auto-correlation functions}
The cross-correlation function of two functions $f(z)$ and $g(z)$ is
defines as
\begin{equation}
 f(z) \otimes g(z) \equiv \intinfty f(z')g(z'-z) dz',
\end{equation}
where $\otimes$ represents correlation.

The correlation function between a same function is called as
auto-correlation, and is defined as
\begin{equation}
 f(z) \otimes f(z) \equiv \intinfty f(z')f(z'-z) dz'.
\end{equation}

\subsubsection*{Wiener-Khintchine's theorem}
The inverse Fourier transform of the power spectrum of a function is the
auto-correlation of the function;
\begin{equation}
 \ift{ \abs{\tilde{f}(k)}^2} = f(z) \otimes f(z).
\end{equation}
This relation between the power spectrum and the auto-correlation is
known as Wiener-Khintchine's theorem.

\section{Scaling and mirroring of a signal}
\subsubsection*{Scaling property of Fourier transform}
The Fourier transform of a signal scaled in $z$ is
\begin{equation}
 \ft{f(az)} = \frac{1}{\abs{a}} \tilde{f}\left(\frac{k}{a}\right).
\end{equation}
Similarly,
\begin{equation}
 \ift{\tilde{f}(bk)} = \frac{1}{\abs{b}}f\left(\frac{z}{b}\right),
\end{equation}
where $a$ and $b$ are scaling factors.

\subsubsection*{Fourier transform of a mirrored signal}
The Fourier transform of a mirrored signal is obtained as
\begin{equation}
 \ft{f(-z)} = \tilde{f}(-k)
  \quad \Leftrightarrow \quad
  \ift{\tilde{f}(-k)} = f(-z).
\end{equation}
This is a special case of the scaling property with a scaling factor of
-1.

\section{Fourier transform of  real function and complex conjugate}
\subsubsection*{Fourier transform of real function}
The Fourier transform spectrum of a real function $r(z)$ is a
mirror-conjugate-symmetric function;
\begin{equation}
 \tilde{r}(k) = \tilde{r^*}(-k)
\end{equation}
where $\tilde{r}(k) = \ft{r(z)}$ and $r(z)$ is a real function.

Similarly, for a real spectrum $\tilde{s}(k)$,
\begin{equation}
 s(z) = s^*(-z)
\end{equation}
where $s(k) = \ift{\tilde{s}(k)}$ and $s(k)$ is a real function.

This relation is exemplified by FD-OCT.
A spectral interferogram is a real function, and hence its Fourier
transform in $z$-domain is mirror-conjugate-symmetric.
(Note that it does not true if additional phase is applied to the
spectral interferogram, as it is frequently done to correct dispersion.)

\subsubsection*{Fourier transform of complex conjugate}
The Fourier transform of the complex conjugate of a function becomes a
mirrored complex conjugate of the original spectrum;
\begin{equation}
 \ft{f^*(z)} = \tilde{f}^*(-k),
\end{equation}
where superscript of $*$ denotes complex conjugate.
Similarly,
\begin{equation}
 \ift{\tilde{f}^*(k)} = f^*(-z).
\end{equation}

This relationship can be derived by resolving $f(z)$ into two real
functions as $f_r(z) + i f_i'(z)$, Fourier transform it, and then
applying the rule for the Fourier transform of a real function.

\section{Parseval's theorem}
Parseval's theorem is the law of energy conservation during Fourier
transform.
\begin{equation}
 \intinfty \abs{f(z)}^2 dz = \frac{1}{2\pi}
  \intinfty{\abs{\tilde{f}(k)}^2} dk
\end{equation}

%\section{Poisson's formula and comb function}
%\begin{equation}
% \mathrm{I\!I\!I}(k)
%\end{equation}

%\section{Window functions and is Fourier transform}
%\subsubsection*{Gauss function}
%\subsubsection*{Hanning function}
%\subsubsection*{Hamming function}
\end{document}